\chapter{Επίλογος}
\label{ch:conclusion}

Στο κεφάλαιο αυτό συνοψίζονται τα βασικά συμπεράσματα της εργασίας και
διατυπώνονται οι κύριες κατευθύνσεις για συνέχεια της μελέτης. Η αποτίμηση
βασίζεται στο ενιαίο πρωτόκολλο του Κεφαλαίου~\ref{ch:experiments} και στα
αποτελέσματα του Κεφαλαίου~\ref{ch:results}.

\section{Συμπεράσματα}

\paragraph{Μεθοδολογική σύνοψη.}
Η εργασία ανέπτυξε ένα ενιαίο και αναπαραγώγιμο \en{pipeline} για
τμηματοποίηση πλακούντα σε \en{3D MRI}, με κοινή προεπεξεργασία, κοινή
διαδικασία εκπαίδευσης/επικύρωσης και ενιαίο τρόπο αναφοράς μετρικών.
Με τον τρόπο αυτό, οι διαφορές μεταξύ μοντέλων ερμηνεύονται κυρίως ως
αρχιτεκτονικές διαφορές και όχι ως τεχνητές αποκλίσεις ρυθμίσεων.

\paragraph{Συγκριτική αξιολόγηση μοντέλων.}
Η συνοπτική κατάταξη της Ενότητας~\ref{sec:results_ranking} και η αναλυτική
συζήτηση της Ενότητας~\ref{sec:results_discussion} οδηγούν στα εξής κύρια ευρήματα:
\begin{itemize}
  \item Τα \en{SegMamba Heavier} και \en{SegResNet Heavier} πέτυχαν τις υψηλότερες τιμές
  επικάλυψης (\en{Dice}$\approx 0.86$), με πολύ μικρή μεταξύ τους διαφορά.
  \item Το \en{SwinUNETR} αποτέλεσε το πιο ανταγωνιστικό
  \en{Transformer}-βασισμένο μοντέλο, αλλά παρέμεινε χαμηλότερα από την ομάδα
  κορυφής.
  \item Το \en{UNETR} λειτούργησε ως σαφής \en{outlier} με αισθητά χαμηλότερη
  επίδοση σε \en{Dice/IoU} στο συγκεκριμένο \en{split}.
  \item Οι παραλλαγές μεγαλύτερης χωρητικότητας (\en{SegResNet Heavier},
  \en{SegMamba Heavier}) έδωσαν μικρό αλλά μετρήσιμο όφελος έναντι των ελαφρύτερων
  εκδόσεων, χωρίς δραστική μεταβολή της συνολικής εικόνας.
\end{itemize}

\paragraph{Συνέπεια ποσοτικής και ποιοτικής ανάλυσης.}
Η ποιοτική αξιολόγηση της Ενότητας~\ref{sec:results_qual} ευθυγραμμίζεται με
τις μετρικές \en{Dice/IoU}. Τα μοντέλα κορυφής παρήγαγαν πιο συνεκτικές μάσκες,
καλύτερη μορφολογική συμφωνία και λιγότερα ψευδοθετικά, ενώ τα μοντέλα με
χαμηλότερες μετρικές εμφάνισαν αστάθειες στα όρια και συχνότερα σφάλματα εκτός στόχου.

\paragraph{Σύνδεση με τη βιβλιογραφία.}
Η σύγκριση της Ενότητας~\ref{sec:results_literature_compare} δείχνει ότι οι
επιδόσεις της παρούσας μελέτης κινούνται σε ανταγωνιστικό εύρος σε σχέση με
πρόσφατες εργασίες τμηματοποίησης πλακούντα. Παράλληλα, η χρήση ενιαίου
πρωτοκόλλου ενισχύει τη σαφήνεια της σύγκρισης μεταξύ οικογενειών αρχιτεκτονικών.

\paragraph{Κλινική συνάφεια και πρακτική αξιοποίηση.}
Τα αποτελέσματα υποστηρίζουν ότι σύγχρονες \en{CNN}- και \en{SSM}-βασισμένες
προσεγγίσεις μπορούν να προσφέρουν αξιόπιστη αυτόματη τμηματοποίηση πλακούντα,
χρήσιμη για ποσοτική ανάλυση και συγκρίσιμες μετρήσεις σε \en{MRI}. Η σταθερότητα
των κορυφαίων μοντέλων αποτελεί θετική ένδειξη για ενσωμάτωση σε
\en{research-oriented} ροές εργασίας.

\paragraph{Περιορισμοί εγκυρότητας.}
Η ερμηνεία των ευρημάτων οφείλει να λάβει υπόψη ότι:
\begin{itemize}
  \item η αξιολόγηση βασίζεται σε \en{validation split} και όχι σε ανεξάρτητο
  \en{test set},
  \item δεν πραγματοποιήθηκε εξωτερική επικύρωση σε δεδομένα άλλου κέντρου ή
  διαφορετικού πρωτοκόλλου,
  \item οι διαφορές στην ομάδα κορυφής είναι μικρές (τρίτη δεκαδική του
  \en{Dice}) και δεν πρέπει να υπερερμηνεύονται.
\end{itemize}

\paragraph{Τελική αποτίμηση.}
Ο βασικός στόχος της διπλωματικής, δηλαδή η τεκμηριωμένη και δίκαιη συγκριτική
αξιολόγηση ετερογενών αρχιτεκτονικών για τμηματοποίηση πλακούντα σε \en{3D MRI},
επιτεύχθηκε. Με τα παρόντα δεδομένα και υπό το ίδιο πρωτόκολλο εκπαίδευσης,
τα \en{SegMamba} και \en{SegResNet} αποτελούν τις ισχυρότερες πρακτικές επιλογές.

\section{Μελλοντικές επεκτάσεις}

\paragraph{Ισχυρότερη εκτίμηση γενίκευσης.}
Κρίσιμη συνέχεια αποτελεί η αξιολόγηση σε ανεξάρτητο \en{test set} και η χρήση
\en{cross-validation}, ώστε τα συμπεράσματα να υποστηρίζονται από πιο σταθερή
στατιστική τεκμηρίωση.

\paragraph{Εξωτερική επικύρωση και \en{domain shift}.}
Απαιτείται έλεγχος των μοντέλων σε δεδομένα από διαφορετικά κέντρα,
\en{scanners} και πρωτόκολλα απεικόνισης, για καλύτερη αποτίμηση ανθεκτικότητας
σε μεταβολές κατανομής.

\paragraph{Μετρικές ορίων και ανάλυση σφαλμάτων.}
Η προσθήκη \en{boundary-sensitive} μετρικών (π.\,χ. \en{HD95}) και η
συστηματική κατηγοριοποίηση αποτυχιών ανά μορφολογικό μοτίβο θα προσφέρουν
πληρέστερη εικόνα πέρα από τις \en{overlap} μετρικές.

\paragraph{Αποδοτικότητα και επιχειρησιακή αξιολόγηση.}
Η καταγραφή χρόνου εκπαίδευσης/\en{inference}, απαιτήσεων \en{VRAM} και
συνολικού κόστους θα βοηθήσει στην επιλογή μοντέλων με ισορροπία ακρίβειας
και υπολογιστικής βιωσιμότητας.

\paragraph{Εκτίμηση αβεβαιότητας και \en{human-in-the-loop}.}
Η ενσωμάτωση χαρτών αβεβαιότητας και στρατηγικών ημι-αυτόματης διόρθωσης
μπορεί να αυξήσει τη διαφάνεια και την αξιοπιστία σε απαιτητικές περιπτώσεις.

\paragraph{Πολυτροπική και πολυ-εργασιακή μοντελοποίηση.}
Πιθανές επεκτάσεις περιλαμβάνουν αξιοποίηση επιπλέον ακολουθιών \en{MRI} όπου
διατίθενται, καθώς και πολυ-εργασιακή μάθηση που συνδυάζει τμηματοποίηση με
παράλληλη πρόβλεψη κλινικών δεικτών.
