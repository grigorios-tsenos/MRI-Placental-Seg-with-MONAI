\chapter{Επίλογος}
\label{ch:conclusion}

Στο κεφάλαιο αυτό συνοψίζονται τα βασικά συμπεράσματα της εργασίας και
διατυπώνονται οι κύριες κατευθύνσεις για συνέχεια της μελέτης. Η αποτίμηση
βασίζεται στο ενιαίο πρωτόκολλο του Κεφαλαίου~\ref{ch:experiments} και στα
αποτελέσματα του Κεφαλαίου~\ref{ch:results}.

\section{Συμπεράσματα}

\paragraph{Μεθοδολογική σύνοψη.}
Η εργασία ανέπτυξε ένα ενιαίο και αναπαραγώγιμο \en{pipeline} για
τμηματοποίηση πλακούντα σε \en{3D MRI}, με κοινή προεπεξεργασία, κοινή
διαδικασία εκπαίδευσης/επικύρωσης και ενιαίο τρόπο αναφοράς μετρικών.

Κατ$'$αυτόν τον τρόπο, οι διαφορές ανάμεσα στα μοντέλα μοντέλων ερμηνεύονται κυρίως ως
αρχιτεκτονικές διαφορές και όχι ως τεχνητές αποκλίσεις ρυθμίσεων.

\paragraph{Συγκριτική αξιολόγηση μοντέλων.}
Η συνοπτική κατάταξη της Ενότητας~\ref{sec:results_ranking} και η αναλυτική
συζήτηση της Ενότητας~\ref{sec:results_discussion} οδηγούν στα εξής κύρια ευρήματα:
\begin{itemize}
  \item Τα \en{SegMamba} και \en{SegResNet} πέτυχαν τις υψηλότερες τιμές
  επικάλυψης (\en{Dice}$\approx 0.86$), με πολύ μικρή μεταξύ τους διαφορά.
  \item Το \en{SwinUNETR} αποτέλεσε το πιο ανταγωνιστικό
  \en{Transformer}-βασισμένο μοντέλο. Παρότι οι μετρικές του είναι ελαφρώς
  χαμηλότερες από την ομάδα κορυφής, στην ποιοτική αξιολόγηση οι διαφορές δεν
  είναι έντονες και η συνολική οπτική συμπεριφορά παραμένει συγκρίσιμη.
  \item Το \en{UNETR} λειτούργησε ως σαφής \en{outlier} με αισθητά χαμηλότερη
  επίδοση σε \en{Dice/IoU} στο συγκεκριμένο \en{split}.
  \item Οι παραλλαγές μεγαλύτερης χωρητικότητας 
  (\en{SegResNet, SegMamba} και \en{SwinUNETR}) έδωσαν αμελητέα βελτίωση, 
  χωρίς να μεταβάλουν ουσιαστικά τη συνολική εικόνα. Ειδικότερα, για  
  το \en{SegResNet Heavier}, ο χρόνος εκτέλεσης αυξήθηκε δυσανάλογα σε 
  σχέση με την αντίστοιχη αύξηση της ακρίβειας.
\end{itemize}

\paragraph{Συνέπεια ποσοτικής και ποιοτικής ανάλυσης.}
Η ποιοτική αξιολόγηση της Ενότητας~\ref{sec:results_qual} ευθυγραμμίζεται με
τις μετρικές \en{Dice/IoU}. Τα μοντέλα κορυφής παρήγαγαν πιο συνεκτικές μάσκες,
καλύτερη μορφολογική συμφωνία και λιγότερα ψευδοθετικά, ενώ τα μοντέλα με
χαμηλότερες μετρικές εμφάνισαν αστάθειες στα όρια και συχνότερα σφάλματα εκτός στόχου.

\paragraph{Σύνδεση με τη βιβλιογραφία.}
Η σύγκριση της Ενότητας~\ref{sec:results_literature_compare} δείχνει ότι οι
επιδόσεις της παρούσας μελέτης κινούνται σε ανταγωνιστικό εύρος σε σχέση με
πρόσφατες εργασίες τμηματοποίησης πλακούντα. Παράλληλα, η χρήση ενιαίου
πρωτοκόλλου ενισχύει τη σαφήνεια της σύγκρισης μεταξύ οικογενειών αρχιτεκτονικών.

\paragraph{Κλινική συνάφεια και πρακτική αξιοποίηση.}
Τα αποτελέσματα υποστηρίζουν ότι σύγχρονες \en{CNN}-, \en{SSM}- και 
\en{Transformer-based} 
προσεγγίσεις μπορούν να προσφέρουν αξιόπιστη αυτόματη τμηματοποίηση πλακούντα,
χρήσιμη για ποσοτική ανάλυση και συγκρίσιμες μετρήσεις σε \en{MRI}. 

\paragraph{Περιορισμοί εγκυρότητας.}
Η ερμηνεία των ευρημάτων οφείλει να λάβει υπόψη ότι:
\begin{itemize}
  \item η αξιολόγηση βασίζεται σε \en{validation split} και όχι σε ανεξάρτητο
  \en{test set},
  \item δεν πραγματοποιήθηκε εξωτερική επικύρωση σε δεδομένα άλλου κέντρου ή
  διαφορετικού πρωτοκόλλου,
  \item οι διαφορές στα αποτελέσματα των καλύτερων είναι μικρές (τρίτη δεκαδική του
  \en{Dice}) και δεν πρέπει να υπερερμηνεύονται.
\end{itemize}

\paragraph{Τελική αποτίμηση.}
Ο βασικός στόχος της διπλωματικής ήταν η τεκμηριωμένη και δίκαιη συγκριτική
αξιολόγηση ετερογενών αρχιτεκτονικών για τμηματοποίηση πλακούντα σε \en{3D MRI}. 
Με τα παρόντα δεδομένα και υπό το ίδιο πρωτόκολλο εκπαίδευσης, ως ισχυρότερες επιλογές αποτελούν
τα \en{SegMamba} και \en{SegResNet}. Παράλληλα, τα \en{SwinUNETR, Attention U-Net}
και \en{DynUNet} εμφανίζουν πολύ κοντινή πρακτική συμπεριφορά, καθώς οι μικρές
διαφορές των μετρικών δεν αποτυπώνονται με σαφήνεια στα ποιοτικά αποτελέσματα.

\section{Μελλοντικές επεκτάσεις}

Κρίσιμη συνέχεια αποτελεί η αξιολόγηση σε ανεξάρτητο \en{test set} και η χρήση
\en{cross-validation}, ώστε τα συμπεράσματα να υποστηρίζονται από πιο σταθερή
στατιστική τεκμηρίωση. Επιπλέον, για καλύτερη αποτίμηση ανθεκτικότητας σε μεταβολές κατανομής
προτείνεται έλεγχος των μοντέλων σε διαφορετικά σύνολα δεδομένων.

Η καταγραφή χρόνου εκπαίδευσης/\en{inference}, απαιτήσεων \en{VRAM} και
συνολικού κόστους θα βοηθήσει στην επιλογή μοντέλων με ισορροπία ακρίβειας
και υπολογιστικής βιωσιμότητας.

Η ενσωμάτωση χαρτών αβεβαιότητας και στρατηγικών ημι-αυτόματης διόρθωσης
μπορεί να αυξήσει τη διαφάνεια και την αξιοπιστία σε απαιτητικές περιπτώσεις.

\paragraph{Επέκταση με \en{fine-tuning} από υπάρχοντα \en{checkpoints}.}
Μια πρακτική κατεύθυνση είναι η αξιοποίηση των καλύτερων
\en{checkpoints} της εργασίας ως σημείο εκκίνησης
για στοχευμένο \en{fine-tuning}. Η προσέγγιση αυτή μπορεί να
επιταχύνει τη σύγκλιση και να υποστηρίξει επεκτάσεις όπως βελτίωση ορίων,
προσαρμογή σε νέα πρωτόκολλα λήψης και συνδυασμό με πρόσθετες εργασίες
τμηματοποίησης.

Παράλληλα, δεδομένου ότι στο τρέχον πρωτόκολλο η επικύρωση έγινε με
\en{foreground cropping} (\en{source\_key=label}), ένα επόμενο βήμα είναι η
επαναξιολόγηση των ίδιων \en{checkpoints} με ροές επικύρωσης που δεν βασίζονται
σε \en{label}-καθοδηγούμενο \en{cropping}, ώστε η διαδικασία να είναι πιο κοντά
σε πραγματικές συνθήκες χρήσης.

\paragraph{Πολυτροπική και πολυ-εργασιακή μοντελοποίηση.}
Πιθανές επεκτάσεις περιλαμβάνουν αξιοποίηση επιπλέον ακολουθιών \en{MRI} όπου
διατίθενται, καθώς και πολυ-εργασιακή μάθηση που συνδυάζει τμηματοποίηση με
παράλληλη πρόβλεψη κλινικών δεικτών.
