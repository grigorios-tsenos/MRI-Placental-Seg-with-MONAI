\chapter{Πειραματική Αξιολόγηση και Αποτελέσματα}
\label{ch:experiments}

Στο κεφάλαιο αυτό παρουσιάζεται η πειραματική διαδικασία που ακολουθήθηκε
για την εκπαίδευση και τη συγκριτική αξιολόγηση διαφορετικών αρχιτεκτονικών
τμηματοποίησης σε \en{3D MRI}, καθώς και τα αντίστοιχα αποτελέσματα.

Η επιλογή των μοντέλων προέκυψε μέσα από συστηματική μελέτη της βιβλιογραφίας:
αρχικά εξετάστηκαν τα θεμελιώδη \en{U-shaped} δίκτυα (\en{U-Net})
\cite{ronneberger2015unet} και στη συνέχεια παραλλαγές/επεκτάσεις όπως το
\en{Attention U-Net} \cite{oktay2018attentionunet} και σύγχρονες \en{U-Net} μορφές
όπως το \en{DynUNet} \cite{isensee2021nnunet}.
Τέλος, αξιολογήθηκαν νεότερες προσεγγίσεις που μοντελοποιούν μακρινές
εξαρτήσεις, όπως τα \en{Transformer-based} μοντέλα
(\en{UNETR} \cite{hatamizadeh2022unetr},
\en{SwinUNETR} \cite{hatamizadeh2022swinunetr})
και \en{State Space-based} μοντέλα
(\en{SegMamba} \cite{xing2024segmamba}, βασισμένο στο \en{Mamba}
\cite{gu2023mamba}).

Κεντρικός στόχος είναι η \textbf{δίκαιη και αναπαραγώγιμη σύγκριση} των
αρχιτεκτονικών. Για τον λόγο αυτό υλοποιήθηκε ένα \textbf{ενιαίο πειραματικό
πλαίσιο} όπου όλες οι αρχιτεκτονικές εκπαιδεύονται με κοινή ροή
(\en{pipeline}) προεπεξεργασίας, δειγματοληψίας \en{patches}, εκπαίδευσης,
αξιολόγησης και καταγραφής μετρικών.
Στην πράξη, κάθε αρχιτεκτονική αντιστοιχεί σε ξεχωριστό \en{notebook},
όμως η δομή του κώδικα παραμένει σταθερή, με κύρια διαφοροποίηση την
αρχικοποίηση του μοντέλου και ελάχιστες αναγκαίες ρυθμίσεις συμβατότητας.

\section{Στόχοι αξιολόγησης και αρχές σύγκρισης}
\label{sec:exp_principles}

Στόχος της πειραματικής αξιολόγησης είναι να αποτυπωθεί με σαφήνεια η
σχετική συμπεριφορά διαφορετικών οικογενειών αρχιτεκτονικών σε τμηματοποίηση
πλακούντα σε \en{3D MRI}. Η σύγκριση σχεδιάστηκε ώστε να είναι δίκαιη, με
κοινό τρόπο αναφοράς αποτελεσμάτων και συνεπείς επιλογές αξιολόγησης, έτσι
ώστε οι παρατηρούμενες διαφορές να αποδίδονται κατά το δυνατόν στην
αρχιτεκτονική.

Η αναφορά βασίζεται κυρίως σε \en{Dice} και \en{IoU}, μετρικές που είναι
κατάλληλες για προβλήματα έντονης ανισορροπίας κλάσεων, ενώ η ερμηνεία δεν
στηρίζεται μόνο στη μέση τιμή αλλά συμπληρώνεται από ποιοτική επιθεώρηση και
σχολιασμό τυπικών αποτυχιών ανά περίπτωση (βλ. Κεφάλαιο~\ref{ch:discussion}).

\section{Υλοποίηση και λογισμικό}
\label{sec:exp_software}

Η υλοποίηση πραγματοποιήθηκε στο \en{PyTorch},
με χρήση του \en{MONAI} \cite{cardoso2022monai}, το οποίο παρέχει έτοιμες
αρχιτεκτονικές, μετασχηματισμούς (\en{transforms}), συναρτήσεις απώλειας,
και εργαλεία αξιολόγησης κατάλληλα για \en{3D} ιατρική τμηματοποίηση.

Όλες οι αρχιτεκτονικές (εκτός του \en{SegMamba}, που ενσωματώθηκε ως
εξωτερική υλοποίηση) χρησιμοποιούν κοινές συνιστώσες του \en{MONAI} για:
\begin{itemize}
  \item (α) ανάγνωση \en{NIfTI} δεδομένων,
  \item (β) προεπεξεργασία/επαύξηση, 
  \item (γ) \en{sliding-window inference},
  \item (δ) υπολογισμό μετρικών.
\end{itemize}

\section{Σύνολο δεδομένων και διαχωρισμός}
\label{sec:exp_data}

Το σύνολο δεδομένων περιλαμβάνει $N=137$ περιστατικά, με ένα ζεύγος αρχείων
\en{\texttt{.nii.gz}} ανά περίπτωση: έναν ογκομετρικό όγκο \en{MRI} και την
αντίστοιχη δυαδική μάσκα τμηματοποίησης πλακούντα.
Ως βασική μονάδα εκπαίδευσης και αξιολόγησης ορίζεται το \en{case}
(δηλαδή ολόκληρος ο \en{3D} όγκος), ώστε ο διαχωρισμός να γίνεται σε επίπεδο
ασθενή/περίπτωσης και να αποφεύγεται διαρροή πληροφορίας.

Ο διαχωρισμός σε σύνολα εκπαίδευσης και επικύρωσης πραγματοποιείται με
σταθερό \en{seed} ($121$) για λόγους αναπαραγωγιμότητας.
Στα πειράματα της παρούσας εργασίας χρησιμοποιήθηκε διαχωρισμός $80/20$,
που αντιστοιχεί περίπου σε $109$ περιστατικά εκπαίδευσης και $28$ επικύρωσης.

\section{Αρχιτεκτονικές και πειραματικά σενάρια}
\label{sec:exp_models}

Αξιολογήθηκαν οι παρακάτω αρχιτεκτονικές:
\begin{itemize}
  \item \en{U-Net} \cite{ronneberger2015unet}
  \item \en{Attention U-Net} \cite{oktay2018attentionunet}
  \item \en{DynUNet} \cite{isensee2021nnunet}
  \item \en{UNETR} \cite{hatamizadeh2022unetr}
  \item \en{SwinUNETR} \cite{hatamizadeh2022swinunetr}
  \item \en{SegResNet} \cite{myronenko2018segresnet}
  \item \en{SegMamba} \cite{xing2024segmamba}
\end{itemize}

\paragraph{Οικογένειες μοντέλων.}
Οι αρχιτεκτονικές που αξιολογούνται καλύπτουν κλασικές \en{CNN} προσεγγίσεις
(\en{U-Net} και παραλλαγές), νεότερες \en{Transformer}-βασισμένες δομές
(\en{UNETR}, \en{SwinUNETR}) και αποδοτικές εναλλακτικές για μοντελοποίηση
μακρινών εξαρτήσεων (\en{state space}-based, \en{SegMamba}). Με αυτόν τον τρόπο
εξετάζεται τόσο η επίδραση ισχυρού τοπικού \en{inductive bias} όσο και η
συμβολή μηχανισμών παγκόσμιου \en{context}.

Για κάθε αρχιτεκτονική εκτελέστηκε μία βασική εκπαίδευση
(\en{baseline run}) με το κοινό \en{pipeline}.
Επιπλέον πραγματοποιήθηκαν ενισχυμένα πειράματα (\en{stronger runs}) για
επιλεγμένα μοντέλα (π.\,χ. \en{SegResNet} και \en{SwinUNETR}), με στόχο να
διερευνηθεί η επίδραση ρυθμίσεων που αυξάνουν την ικανότητα του δικτύου ή/και
βελτιώνουν τη σταθερότητα σύγκλισης (π.\,χ. μεταβολές σε
\en{feature size, window size}, βάθος, κ.\,ά.).
Οι αποκλίσεις αυτές από τη βασική ρύθμιση καταγράφονται ρητά στις αντίστοιχες
υποενότητες αποτελεσμάτων.

\section{Παρουσίαση αποτελεσμάτων}
\label{sec:exp_results}

Τα αποτελέσματα παρουσιάζονται σε δύο επίπεδα:
\begin{itemize}
  \item \textbf{Ποσοτικά:} μετρικές \en{Dice/IoU} και απώλειες
        (\en{train/val loss}), με σύγκριση μεταξύ μοντέλων.
  \item \textbf{Ποιοτικά:} ενδεικτικές οπτικοποιήσεις προβλέψεων/σφαλμάτων
        σε αντιπροσωπευτικά περιστατικά.
\end{itemize}

Ο Πίνακας~\ref{tab:main_results} συνοψίζει τη βασική επίδοση ανά μοντέλο στο
\en{validation set}. Η αναλυτική ερμηνεία, συγκριτική συζήτηση και
\en{error analysis} παρουσιάζονται στο Κεφάλαιο~\ref{ch:discussion}.

\begin{table}[h]
\centering
\caption{Σύνοψη αποτελεσμάτων ανά αρχιτεκτονική στο \en{validation set}.}
\label{tab:main_results}
\begin{tabular}{lccc}
\hline
\textbf{Μοντέλο} & \en{\textbf{Dice}} & \en{\textbf{IoU}} & \en{\textbf{Val Loss}} \\
\hline
\en{U-Net}            &  &  &  \\
\en{Attention U-Net}  &  &  &  \\
\en{DynUNet}          &  &  &  \\
\en{UNETR}            &  &  &  \\
\en{SwinUNETR}        &  &  &  \\
\en{SegResNet}        &  &  &  \\
\en{SegMamba}         &  &  &  \\
\hline
\end{tabular}
\end{table}