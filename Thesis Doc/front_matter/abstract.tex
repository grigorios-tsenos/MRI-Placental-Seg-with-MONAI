\begin{abstract}
Η αυτόματη τμηματοποίηση του πλακούντα σε \en{MRI} αποτελεί θεμελιώδες βήμα
για την ποσοτική ανάλυση της κύησης και τη συστηματική αξιοποίηση πληροφοριών
ιατρικής απεικόνισης. Η διαδικασία είναι ιδιαίτερα απαιτητική, λόγω έντονης
μεταβλητότητας στη μορφολογία και στην εμφάνιση του οργάνου, ασαφών ορίων, και
της περιορισμένης διαθεσιμότητας επαρκώς επισημασμένων δεδομένων.

Η παρούσα διπλωματική εργασία πραγματοποιεί συστηματική συγκριτική αξιολόγηση
σύγχρονων αρχιτεκτονικών βαθιάς μάθησης για \en{3D} τμηματοποίηση πλακούντα.
Αναπτύχθηκε ενιαίο, αναπαραγώγιμο πειραματικό πλαίσιο σε \en{MONAI/PyTorch},
το οποίο ενοποιεί προεπεξεργασία, δειγματοληψία, επαύξηση, εκπαίδευση και
αξιολόγηση, ώστε η σύγκριση να είναι δίκαιη και τεκμηριωμένη.

Η μελέτη καλύπτει αρχιτεκτονικές τύπου \en{U-Net}, \en{Transformer}-βασισμένες
προσεγγίσεις και \en{state-space}-βασισμένα μοντέλα, με αποτίμηση μέσω
\en{Dice/IoU} και ποιοτικής επιθεώρησης των παραγόμενων μασκών. Τα ευρήματα
αναδεικνύουν ότι η αυστηρή τυποποίηση του \en{pipeline} αποτελεί κρίσιμο
παράγοντα για αξιόπιστα συμπεράσματα και ότι διαφορετικές οικογένειες
αρχιτεκτονικών παρουσιάζουν διακριτά πλεονεκτήματα ως προς τη σταθερότητα, την
ποιότητα ορίων και τη συνολική συμπεριφορά τμηματοποίησης.

Συνολικά, η εργασία προσφέρει ένα σαφές πλαίσιο αναφοράς για την επιλογή
μοντέλων τμηματοποίησης πλακούντα σε \en{MRI}. Παράλληλα, υπογραμμίζει την ανάγκη
για περαιτέρω διερεύνηση της γενίκευσης σε ανεξάρτητα σύνολα δεδομένων, για
εξωτερική επικύρωση, καθώς και για ενσωμάτωση τεχνικών εκτίμησης αβεβαιότητας.

   \begin{keywords}
   \en{MRI}, Πλακούντας, Τμηματοποίηση ιατρικών εικόνων, Βαθιά Μάθηση, \en{MONAI},
   Συγκριτική αξιολόγηση, \en{U-Net}, \en{Transformers}, \en{State Space Models},
   Αναπαραγωγιμότητα
   \end{keywords}
\end{abstract}

\begin{abstracteng}
\tl{Automatic placenta segmentation in \en{MRI} is a fundamental step for
quantitative pregnancy analysis and for the systematic use of medical imaging
information. The task is particularly challenging due to strong variability in
organ morphology and appearance, ambiguous boundaries, and the limited
availability of sufficiently annotated data.}

\tl{This thesis performs a systematic comparative evaluation of modern deep
learning architectures for \en{3D} placenta segmentation. A unified,
reproducible experimental framework was developed in \en{MONAI/PyTorch},
integrating preprocessing, sampling, augmentation, training, and evaluation, so
that the comparison is fair and well documented.}

\tl{The study covers \en{U-Net}-type architectures, \en{Transformer}-based
approaches, and \en{state-space}-based models, assessed through \en{Dice/IoU}
and qualitative inspection of the produced masks. The findings highlight that
strict \en{pipeline} standardization is a critical factor for reliable
conclusions, and that different architecture families exhibit distinct strengths
in stability, boundary quality, and overall segmentation behavior.}

\tl{Overall, the thesis provides a clear reference framework for selecting
placenta segmentation models in \en{MRI}. At the same time, it underlines the
need for further investigation of generalization on independent datasets,
external validation, and the integration of uncertainty estimation techniques.}

   \begin{keywordseng}
    \tl{MRI, Placenta, Medical image segmentation, Deep Learning, MONAI,
    Comparative evaluation, U-Net, Transformers, State Space Models,
    Reproducibility}
   \end{keywordseng}

\end{abstracteng}
