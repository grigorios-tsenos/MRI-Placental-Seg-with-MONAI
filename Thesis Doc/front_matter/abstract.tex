\begin{abstract}
Η αυτόματη τμηματοποίηση πλακούντα σε \en{MRI} αποτελεί κρίσιμο βήμα για
ποσοτική ανάλυση της κύησης, καθώς μπορεί να υποστηρίξει πιο αντικειμενικές
μετρήσεις και συγκρίσεις μεταξύ περιστατικών. Το πρόβλημα παραμένει απαιτητικό
λόγω μεταβλητότητας στη μορφολογία, στην ένταση και στα όρια του οργάνου, καθώς
και λόγω της σχετικά περιορισμένης διαθεσιμότητας επισημασμένων ιατρικών δεδομένων.

Αντικείμενο της παρούσας διπλωματικής είναι η συστηματική συγκριτική αξιολόγηση
σύγχρονων αρχιτεκτονικών βαθιάς μάθησης για \en{3D} τμηματοποίηση πλακούντα.
Υλοποιήθηκε ενιαίο και αναπαραγώγιμο πειραματικό πλαίσιο σε \en{MONAI/PyTorch},
ώστε οι διαφορές στην απόδοση να αποδίδονται κυρίως στην αρχιτεκτονική και όχι
σε ασυνεπείς ρυθμίσεις εκπαίδευσης. Μελετήθηκαν μοντέλα \en{U-Net}-οικογένειας
(\en{U-Net}, \en{Attention U-Net}, \en{DynUNet}), \en{Transformer}-βασισμένες
προσεγγίσεις (\en{UNETR}, \en{SwinUNETR}) και \en{state-space}-βασισμένα μοντέλα
(\en{SegMamba}), με αξιολόγηση μέσω \en{Dice} και \en{IoU}.

Στο εξεταζόμενο \en{validation split}, τα καλύτερα αποτελέσματα επιτεύχθηκαν από
το \en{SegMamba Heavier} (\en{Dice}=0.8606, \en{IoU}=0.7566), πολύ κοντά στο
\en{SegResNet Heavier} (\en{Dice}=0.8601). Το \en{SwinUNETR} ήταν το πιο ανταγωνιστικό
\en{Transformer}-βασισμένο μοντέλο (\en{Dice}=0.849), ενώ το \en{UNETR} παρουσίασε
τη χαμηλότερη επίδοση (\en{Dice}=0.772). Η ποιοτική αξιολόγηση ευθυγραμμίζεται με
τις ποσοτικές μετρικές, αναδεικνύοντας πιο συνεκτικές μάσκες και λιγότερα
ψευδοθετικά στα μοντέλα κορυφής.

Συνολικά, η εργασία δείχνει ότι αποδοτικές \en{CNN}- και \en{SSM}-βασισμένες
αρχιτεκτονικές αποτελούν ισχυρές επιλογές για τμηματοποίηση πλακούντα σε \en{MRI}
υπό ενιαίο πρωτόκολλο εκπαίδευσης. Παράλληλα, αναδεικνύεται ανάγκη για
ισχυρότερη εκτίμηση γενίκευσης μέσω ανεξάρτητων συνόλων ελέγχου και εξωτερικής
επικύρωσης.

   \begin{keywords}
   \en{MRI}, Πλακούντας, Τμηματοποίηση ιατρικών εικόνων, Βαθιά Μάθηση, \en{MONAI},
   \en{U-Net}, \en{SegResNet}, \en{SegMamba}, \en{SwinUNETR}, \en{UNETR},
   \en{Dice}, \en{IoU}
   \end{keywords}
\end{abstract}

\begin{abstracteng}
\tl{Placenta segmentation in MRI is a key step for quantitative pregnancy
analysis, since it can support more objective measurements and comparisons
across cases. The task remains challenging due to variability in morphology,
intensity, and organ boundaries, as well as the limited availability of
annotated medical data.}

\tl{This thesis performs a systematic comparative evaluation of modern deep learning
architectures for \en{3D} placenta segmentation. A unified and reproducible
\en{MONAI/PyTorch} framework was implemented, so performance differences can be
attributed primarily to architecture rather than inconsistent training settings.
The study includes \en{U-Net}-family models (\en{U-Net}, \en{Attention U-Net},
\en{DynUNet}), \en{Transformer}-based approaches (\en{UNETR}, \en{SwinUNETR}),
and \en{state-space}-based models (\en{SegMamba}), evaluated with \en{Dice} and
\en{IoU}.}

\tl{On the evaluated validation split, the top result was achieved by
\en{SegMamba Heavier} (\en{Dice}=0.8606, \en{IoU}=0.7566), closely followed by
\en{SegResNet Heavier} (\en{Dice}=0.8601). \en{SwinUNETR} was the most competitive
\en{Transformer}-based model (\en{Dice}=0.849), whereas \en{UNETR} showed the
lowest performance (\en{Dice}=0.772). Qualitative inspection is consistent with
the quantitative metrics, showing cleaner masks and fewer false positives for
the top-performing models.}

\tl{Overall, the results indicate that efficient CNN- and SSM-based
architectures are strong choices for placenta segmentation in \en{MRI} under a
common training protocol. At the same time, stronger generalization assessment
through independent test sets and external validation remains an important next step.}

   \begin{keywordseng}
    \tl{Placenta, MRI, Medical image segmentation, Deep Learning, MONAI,
    U-Net, SegResNet, SegMamba, SwinUNETR, UNETR, Dice, IoU}
   \end{keywordseng}

\end{abstracteng}
